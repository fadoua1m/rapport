\chapter*{Introduction générale}
\addcontentsline{toc}{chapter}{Introduction générale} % to include the introduction to the table of content
\markboth{Introduction générale}{} 
À l’ère de la transformation numérique, les entreprises cherchent de plus en plus à automatiser leurs processus internes afin d’améliorer leur efficacité et d’optimiser la gestion des ressources humaines. Parmi ces processus, la gestion des demandes de congés et l’accès à la réglementation interne représentent des tâches répétitives et chronophages, aussi bien pour les employés que pour les services RH.

Les avancées récentes dans le domaine de l’intelligence artificielle et du traitement du langage naturel offrent aujourd’hui des solutions innovantes pour répondre à ces besoins. L’un des outils les plus utilisés dans ce contexte est le chatbot, un agent conversationnel capable de comprendre les requêtes des utilisateurs et de fournir des réponses instantanées. Grâce à l’intégration de modèles d’intelligence artificielle et de techniques d’enrichissement contextuel telles que le Retrieval-Augmented Generation (RAG), ces chatbots deviennent de véritables assistants capables d’interagir avec les données internes de l’entreprise afin de fournir des informations précises et pertinentes.

Ce projet de fin d’études a pour objectif la conception et le déploiement d’un chatbot RH destiné à automatiser les demandes de congés et à assister les employés dans la consultation de la réglementation interne. Le développement de ce projet repose sur l’exploitation de l’intelligence artificielle et sur l’intégration d’une architecture RAG, permettant de combiner la recherche d’informations dans une base documentaire à la génération de réponses cohérentes et adaptées au contexte.

Le rapport est structuré en quatre chapitres principaux. Le premier chapitre présente le contexte général du projet, l’entreprise d’accueil, la problématique identifiée et les objectifs à atteindre. Le deuxième chapitre traite de l’étude de l’existant et de la conception du chatbot, en détaillant les choix technologiques et l’architecture adoptée. Le troisième chapitre est consacré à la réalisation technique, dans lequel sont exposées les étapes de développement, les outils utilisés et l’intégration du système dans l’environnement de l’entreprise. Le quatrième chapitre présente les résultats obtenus, une évaluation des performances et les perspectives d’amélioration envisagées. Le rapport se termine par une conclusion générale qui résume les principaux apports de ce travail et met en avant les perspectives futures pour l’évolution du projet.