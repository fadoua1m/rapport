\chapter{Contexte général}
\section*{Introduction}
Dans ce premier chapitre, nous présenterons l’entreprise d’accueil, en décrivant son domaine d’activité et son organisation. Nous analyserons ensuite le contexte général de l’application, mettrons en évidence la problématique rencontrée et détaillerons la solution proposée ainsi que les objectifs poursuivis par ce projet.

% Une section

% Exemple d'une section qui porte une référence à une bibliographie
% NB: il faut bien suivre le syntaxe pour ne pas tomber dans le cas où il y a une référence dans la table des matières.
\section{Organisme d'accueil }
Astrolab est une entreprise spécialisée dans le développement informatique. Fondée en 2015, elle intervient dans le développement et la conception de logiciels et dispose de filiales à Doha, Paris et Sousse, en Tunisie. L’équipe de l’entreprise est composée d’experts motivés, proactifs et enthousiastes, spécialisés dans le conseil aux entreprises, le développement d’applications web et mobiles, le design graphique ainsi que la conception d’interfaces utilisateur et l’expérience utilisateur (UI/UX).Le logo de l’entreprise est présent dans la
Figure \ref{fig:logo_tt}. 

% On peut ajouter une figure en utilisant le syntaxe suivant:
\begin{figure}[htpb]
\centering
\frame{\includegraphics[width=0.2\columnwidth]{Logo_Entreprise}}
\caption{Logo Entreprise}
\label{fig:logo_tt}
\end{figure}


\section{Problématique}
Le département des ressources humaines gère de nombreuses tâches répétitives, parmi lesquelles figurent la gestion des demandes de congés, le suivi des absences, l’administration des dossiers du personnel et la diffusion des informations sur la réglementation interne. Ces activités consomment beaucoup de temps et peuvent être sources d’erreurs. Il devient donc nécessaire de mettre en place un système automatisé capable de traiter ces demandes et de fournir des informations fiables, afin d’alléger la charge administrative et d’améliorer l’efficacité du service.

\section{Solution proposée}
La solution proposée pour ce projet consiste à concevoir et déployer un chatbot RH capable d’automatiser la gestion des demandes de congés et de fournir des réponses aux questions relatives à la réglementation interne. Ce système repose sur l’intelligence artificielle et l’architecture Retrieval-Augmented Generation (RAG), permettant de rechercher des informations dans une base documentaire et de générer des réponses précises et adaptées au contexte. L’objectif est d’alléger la charge administrative du service RH, de réduire les erreurs liées aux tâches répétitives et d’améliorer l’efficacité et la satisfaction des utilisateurs.


\section{Choix méthodologique}
Une méthodologie claire et structurée est essentielle pour assurer le succès du projet, respecter les délais et répondre aux exigences fixées. Pour ce projet, nous avons choisi d’adopter la méthode Scrum, une approche agile reconnue pour sa flexibilité, sa collaboration et sa capacité à s’adapter aux changements tout au long du développement. Cette méthodologie permet d’organiser le travail en sprints, de prioriser les fonctionnalités et d’assurer un suivi régulier de l’avancement, garantissant ainsi la réussite du projet et la qualité du chatbot RH développé.
\begin{figure}[htpb]
\centering
{\includegraphics[width=0.8\columnwidth]{f.png}}
\caption{Processus de la methodologie Scrum}
\label{fig:logo_tt}
\end{figure}
\section*{Conclusion}
   Ce chapitre a permis de présenter l’entreprise d’accueil, d’analyser le contexte général du projet, de mettre en évidence la problématique et de détailler la solution proposée ainsi que la méthodologie choisie pour sa réalisation.